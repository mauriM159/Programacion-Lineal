\documentclass{article}

\usepackage[spanish]{babel}
\usepackage{amsmath}
\usepackage[utf8]{inputenc}

\title{Método Simplex}
\author{Mauritania Meneses}

\begin{document}
\maketitle

\section{Introducción}
\label{sec:introduccion}

El método simplex es un algoritmo para resolver
problemas de progamación lineal.Fue inventado por George Dantzing en
el año 1947
\section{Ejemplo}

\label{sec:ejemplo}
Mostremos la aplicacion del método simplex
\begin{equation*}
  \begin{aligned}
    \text{Maximizar} \quad & 2x1+x2\\
    
    \text{sujeto a}\quad &
    
    \quad & \begin{aligned}
     x1+x2 & \geq 1\\  
     3x1+4x2 &\leq 12\\
     x1-x2 &\leq 2\\
     -2x1+x2&\leq 2\\
      
      x1,x2 &\geq 0
    \end{aligned}
  \end{aligned}
\end{equation*}
Para obtener la forma simplex añadimos una variable

\begin{equation*}
  \begin{aligned}
    \text{Maximizar} \quad & 2x1+x2\\
    
    \text{sujeto a}\quad &
    
    \quad & \begin{aligned}
     x1+x2+x3 & \geq 1\\  
     3x1+4x2+x4 &\leq 12\\
     x1-x2+x5 &\leq 2\\
     -2x1+x2+x6&\leq 2\\
     x1,x2,x2,x3,x4,x5,x6 &\geq 0
    \end{aligned}
  \end{aligned}
\end{equation*}

a continuacion obtenemos un \emph{tablero simplex} despejando las
variables de holgura
\begin{equation*}
  \begin{aligned}
    \text{Maximizar} \quad & 2x1+x2\\
    
    \text{sujeto a}\quad &
    
    \quad & \begin{aligned}
    x3&=- x1-x2 \\  
    x4&= 3x1+4x2 \\
    x5&=- x1+x2 \\
    x6&=2x1-x2\\
     x1,x2,x2,x3,x4,x5,x6 &\geq 0

     \hline
     z&=
   \end{aligned}
  \end{aligned}
\end{equation*}

\end{document}
